\section{Decorrelation by Whiteninig}


% --- SLIDE 6: THE TRANSFORMATION WORKFLOW (ANIMATED) ---
\begin{frame}{Implementation Workflow (Step-by-Step)}
    \begin{columns}[c]
        % Left Column: The Steps (Text)
        \begin{column}{0.55\textwidth}
            \small
            \textbf{Standard Operating Procedure (SOP):}

            % [<+->] automatically uncovers items one by one (1, then 2, then 3...)
            \begin{enumerate}[<+->]
                \setlength\itemsep{0.5em}

                \item \textbf{Decompose:} Compute Eigen-decomposition of Covariance Matrix $\hat{\vect{\Sigma}} = \vect{P} \vect{\Lambda} \vect{P}^T$.

                \item \textbf{Transform (Whitening):} Rotate data to eliminate correlation.
                % The box appears with item 2
                \begin{equation*}
                    \boxed{\vect{Z}_i = \vect{\Lambda}^{-1/2}\vect{P}^T(\vect{x}_i - \bar{\vect{x}})}
                \end{equation*}

                \item \textbf{Estimate:} Apply simple Product Kernel on the whitened data $\vect{Z}$.

                \item \textbf{Reconstruct:} Adjust density scale to original space (Jacobian determinant).
                \[ \hat{f}_{\vect{X}}(\vect{x}) = \hat{f}_{\vect{Z}}(\vect{z}) \cdot \alert{|\hat{\vect{\Sigma}}|^{-1/2}} \]
            \end{enumerate}
        \end{column}

        % Right Column: The Image (Static)
        % We keep the image visible from the start so they can reference it while reading steps
        \begin{column}{0.45\textwidth}
            \centering
            \includegraphics[width=0.95\linewidth, height=6cm, keepaspectratio]{whitening_transform}
            \vspace{0.2em}
            \captionof{figure}{\scriptsize $\vect{X}$ (Correlated) $\to$ $\vect{Z}$ (Spherical)}
        \end{column}
    \end{columns}
\end{frame}

% --- SLIDE 7: PERFORMANCE COMPARISON (ANIMATED) ---
\begin{frame}{Performance Comparison}

    \begin{columns}[c]
        % Left Column: Image (Always visible to set the context)
        \begin{column}{0.6\textwidth}
            \centering
            \includegraphics[height=4.0cm, keepaspectratio]{white_mkde_vs_native}
            \vspace{-0.2em}
            \captionof{figure}{\scriptsize Naive (Left) vs. Whitened (Right)}
        \end{column}

        % Right Column: Text Observations (Revealed sequentially)
        \begin{column}{0.4\textwidth}
            \small
            \textbf{Key Observations:}
            \begin{itemize}
                \setlength\itemsep{0.5em}

                % Click 2: Reveal the Failure analysis
                \item<2-> \textbf{Naive Approach:}
                \textcolor{red}{\textbf{Fail.}} Constrained to axis-aligned shapes. Misses diagonal correlations.

                % Click 3: Reveal the Success analysis
                \item<3-> \textbf{Whitened MKDE:}
                \textcolor{darkgreen}{\textbf{Success.}} Contours adapt perfectly to the data's orientation.
            \end{itemize}
        \end{column}
    \end{columns}

    \vspace{1em}

    % Click 4: Reveal the Final Verdict (The "Takeaway" message)
    \onslide<4->{
        \begin{alertblock}{Final Verdict}
            \centering \footnotesize
            Whitening enables \textbf{Product Kernels} to model complex correlations with $O(n)$ efficiency.
        \end{alertblock}
    }
\end{frame}


% --- SLIDE 5: THE MATHEMATICAL SHORTCUT (ANIMATED) ---
\begin{frame}{Generalized Estimator (Theory)}
    Instead of performing manual transformation steps, we can mathematically express the estimator directly using the **Mahalanobis Distance**:

    % 1. The Equation is always visible, but parts light up sequentially
    \begin{block}{General Multivariate KDE Equation (10.44)}
        \begin{equation}
             \hat{f}(\vect{x}) = \frac{\alert<2>{|\hat{\vect{\Sigma}}|^{-1/2}}}{nh^p} \sum_{i=1}^{n} K \left( \frac{\alert<3>{(\vect{x} - \vect{X}_i)^T \hat{\vect{\Sigma}}^{-1} (\vect{x} - \vect{X}_i)}}{h} \right)
        \end{equation}
    \end{block}

    \vspace{1em}
    % 2. Columns appear strictly when their corresponding math term is highlighted
    \begin{columns}[t]
        % Left Column: Appears on Click 2 (Matches |\Sigma|^{-1/2})
        \begin{column}{0.48\textwidth}
            \onslide<2->{
                \textbf{\alert<2>{1. Volume Correction:}}
                \begin{itemize}
                    \item The term $|\hat{\vect{\Sigma}}|^{-1/2}$ adjusts the density scale.
                    \item Ensures the total probability integrates to 1 after "stretching" space.
                \end{itemize}
            }
        \end{column}

        % Right Column: Appears on Click 3 (Matches Mahalanobis Distance)
        \begin{column}{0.48\textwidth}
            \onslide<3->{
                \textbf{\alert<3>{2. Shape Adaptation:}}
                \begin{itemize}
                    \item Uses \textbf{Mahalanobis Distance} instead of Euclidean.
                    \item Captures the correlation structure (orientation) of the data.
                \end{itemize}
            }
        \end{column}
    \end{columns}
\end{frame}